% This text is proprietary.
% It's a part of presentation made by myself.
% It may not used commercial.
% The noncommercial use such as private and study is free
% Sep. 2005 
% Author: Sascha Frank 
% University Freiburg 
% www.informatik.uni-freiburg.de/~frank/

\documentclass{beamer}
\usepackage{multicol}
\usepackage{amsmath}

\begin{document}
\title{Chapter 11 Number-Theoretic Algorithms}   
\author{Jason Pearson and Sam Demorest} 
\date{\today} 

\frame{\titlepage} 

\frame{
	\begin{multicols}{1}
	\frametitle{Overview}\tableofcontents
	\end{multicols}
} 


\section{Number Theory Review} 

\frame{\frametitle{Composite and Prime Numbers} 
Composite Numbers have a divisor other than itself and one. \newline
For example 4$\mid$20 means that 20 = 5 * 4 \newline
The divisors of 12 are 1,2,3,4,6 and 12 \newline
Prime numbers have no divisors but 1 and itself \newline
First 10 Primes \newline
2, 3, 5, 7, 11, 13, 17, 19, 23, 29
}

\section{Greatest Common Divisor}
\frame{\frametitle{Greatest Common Divisor}
	If h$\mid$m and h$\mid$n then h is called a common divisor\newline
	A common divisor is a number that is a factor of both numbers \newline
	The greatest common divisor is the largest factor for both numbers \newline	
	This is denoted gcd(n,m) \newline
	For example gcd(12,15) = 3 \newline \newline
	For any two integers n and m where m $\ne$ 0 the quotient is given by \newline
	$q = \lfloor n / m \rfloor$\newline
	The remainder r of dividing n by m is given by \newline
	$r = n - qm$\newline
}

\frame{\frametitle{Greatest Common Divisor (cont)}
	Let n and m be integers, not both 0 and let\newline
	d = min \{ in + jm such that i,j $\in$ Z and in + jm $>$ 0\}\newline
	That is, d is the smallest positive linear combination of n and m \newline
	For example we know gcd(12, 8) = 4,\newline the smallest linear combination is \newline
	$4 = 3(12) + (-4)8$
	\newline \newline
	Now suppose we have n $\ge$ 0 and m $>$ 0 and r = n mod(m) then \newline
	gcd(n , m) = gcd(m , r)\newline
	so gcd(64 , 24) = gcd(24, 16)\newline
	\hspace*{2.45cm} = gcd(16, 8) \newline
	\hspace*{2.45cm} = gcd(8, 0) \newline
	\hspace*{2.45cm} = 8 \newline
}


\frame{\frametitle{Least Common Multiple}
	For n and m where they are both nonzero, the least common multiple is denoted lcm(n,m)\newline
	For example lcm(6,9) = 18 because 6$\mid$18 and 9 $\mid$18\newline
	The lcm(n,m) is a product of primes that are common to m and n, where the power of each prime in the product is the larger of its orders in n and m\newline
	So $12={2}^{2}×{3}^{1}$ and $45={3}^{2}×{5}^{1}$\newline
	so lcm(12,45) = ${2}^{2}×{3}^{2}×{5}^{1}$ = 180
}

\frame{\frametitle{Prime Factorization}
	Two integers are relatively prime because the gcd of them is 1 \newline
	For example gcd(12, 25) = 1 so they are relatively prime \newline
	If h and m are relatively prime and h divides nm, then h divides m.\newline
	That is gcd(h,m) = 1 and h$\mid$nm implies h$\mid$n \newline
}

\frame{\frametitle{Prime Factorization (cont)}
	Every integer X $>$ 1 can be written as a unique product of primes \newline
	That is X = p$_{1}^{k_{1}}$ * p$_{2}^{k_{2}}$ * ... * p$_{n}^{k_{n}}$  \newline
	Where p$_{1}<$p$_{2}<$...p$_{n}$ and this representation of n is unique \newline
	Example being 22,275 = 3$^{4}$ * 5$^{2}$ * 11 \newline \newline
	To solve gcd(3,185,325, 7,276,500) we know \newline
	3,185,325 = ${3}^{4}{5}^{2}{11}^{2}{13}^{1}$\newline
	7,276,500 = ${2}^{2}{3}^{3}{5}^{3}{7}^{2}{11}^{1}$\newline
	We then take the common divisors and take the lower power to create the gcd \newline
	so gcd(3,185,325, 7,276,500) = ${3}^{3}{5}^{2}{11}^{1}$ = 7,425
}


\section{Euclid's Algorithm}

\frame{\frametitle{Euclid's Algorithm}
	Euclid's Algorithm gives us a straight forward way to find the gcd of two numbers \newline
	int gcd(int n, int m)\newline
	\{\newline
	\hspace*{1cm}if(m == 0)\newline
	\hspace*{2cm} return n;\newline
	\newline
	\hspace*{1cm}else\newline
	\hspace*{2cm} return gcd(m, n mod m);\newline
	\}
}

\frame{\frametitle{Extension to Euclid's Algorithm}
	void Euclid (int n, int m, int gcd, int i, int j)\{ \newline
	\hspace*{1cm}if (m == 0) \{\newline
	\hspace*{2cm}gcd = n; i = 1; j = 0;\newline
	\hspace*{1cm}\}\newline
	\hspace*{1cm}else \{\newline
	\hspace*{2cm}int iprime, jprime, gcdprime;\newline
	\hspace*{2cm}Euclid (m, n mod m, gcdprime, iprime, jprime);\newline
	\hspace*{2cm}gcd = gcdprime;\newline
	\hspace*{2cm}i = jprime;\newline
	\hspace*{2cm}j = iprime -$ \lfloor n / m \rfloor$ jprime ;\newline
	\hspace*{1cm}\}\newline
	\}
}

\frame{\frametitle{Why Use the Other Algorithm?}
	This other algorithm will give us integers i and j as well \newline
	So, gcd = in + jm \newline
	For Example Euclid(42, 30, gcd, i, j) outputs \newline
	gcd = 6, i = -2 and j = 3 \newline
	6 = -2(42) + 3(30) \newline
}

\frame{\frametitle{Proof Extended Algorithm}
	Induction Base: In the last recursive call m = 0, which means gcd(n, m) = n\newline
	Since the values of i and j are assigned values 1 and 0 respectively we have \newline
	$i×n+j×m=1×n+0×m=n=gcd\left(n,m\right)$	\newline \newline
	Induction Hypothesis: Assume in the kth recursive call the values determined for i and j are such that\newline
	gcd(n,m) = in + mj \newline
	Then the values returned by that call for i' and j' are values such that \newline
	gcd(m, n mod m) = i'm + j'n mod m \newline
}

\frame{\frametitle{Proof Extended Algorithm (cont)}
	Induction Step: We have for the (k - 1)st call that\newline
	in + mj = j'n + (i' - $\lfloor n / m \rfloor$ j')m\newline
	\hspace*{1.7cm}= i'm + j'(n - $\lfloor n / m \rfloor$m)\newline
	\hspace*{1.7cm}= i'm + j'n mod m\newline
	\hspace*{1.7cm}= gcd(m, n mod m)\newline
	\hspace*{1.7cm}= gcd(n,m) \newline
	The second to last equality is due to the induction hypothesis\newline	
}

\section{Modular Arithmetic}

\frame{\frametitle{Group Theory}
	A closed binary operation * on a set S is a rule for combining two elements of S to yield another element of S. \newline
	This operation must be associative\newline
	Must have an identity element for each element in S\newline
	For each element in S there must exist an inverse for that element \newline
	For example with integers $\in$ Z with addition constitute a group. \newline
	The identity element is 0 and the inverse of a is -a\newline
	A group is said to be finite if S contains a finite number of elements\newline
	A group is said to be commutative (or abelian) if for all a, b $\in$ S \newline
	$a\ast b=b\ast a$
}

\frame{\frametitle{Congruency Modulo n}
	Let m and k be integers and n be a positive integer. If n$\mid$(m - k) we say m is congruent to k modulo n, and this is written by \newline
	$m\equiv k\mathrm{mod}n$ \newline
	For Example \newline
	Since 5$\mid$(33 - 18), $33\equiv 18\mathrm{mod}5$ \newline \newline
	The integers 2, 5, 9 are pairwise prime and \newline
	$184\equiv 4\mathrm{mod}2$\newline
	$184\equiv 4\mathrm{mod}5$\newline
	$184\equiv 4\mathrm{mod}2$\newline
	Since $2×5×9=90$ this implies $184\equiv 4\mathrm{mod}90$\newline
	Congruency modulo n is an equivalence relation on the set of all integers.
	
}


\frame{\frametitle{Equivalence Class Modulo n Containing m}
	The set of all integers congruent to m modulo n is called the equivalence class modulo n containing m\newline
	For example the equivalence class modulo 5 containing 13 is \newline
	$\left\{\dots ,-7,-2,3,8,13,18,23,28,33,\dots \right\}$\newline
	Equivalence classes modulo n containing m are represented by $[m]_{n}$\newline
	So for our previous example we would represent it by $[3]_{5}$\newline\newline
	The set of all equivalence classes modulo n is denoted $\textbf{Z}_{n}$\newline
	$\textbf{Z}_{n} = \{[0]_{n},[1]_{n},...,[n - 1]_{n}\}$ \newline
	Example of Addition using $\textbf{Z}_{5} = \{[0]_{5},[1]_{5},[2]_{5},[3]_{5},[4]_{5}\}$\newline
	$[2]_{5} + [4]_{5} = [6]_{5} = [1]_{5}$\newline
	For every positive integer n, $(\textbf{Z}_{n}, +)$ is a finite commutative group\newline
	Every element has an additive inverse so we know the identity element is $[0]_{n}$
}

\frame{\frametitle{Equivalence Class Modulo n Containing m (cont)}
	Using $\textbf{Z}_{5} = \{[0]_{5},[1]_{5},[2]_{5},[3]_{5},[4]_{5}\}$\newline
	For multiplication $[2]_{5} * [4]_{5} = [8]_{5} = [3]_{5}$\newline
	This isn't always the case though because not every element in $(\textbf{Z}_{n}, ×)$ has a multiplicative inverse\newline 
	For example we consider $\textbf{Z}_{9}$ \newline
	Suppose $[6]_{9}$ has a multiplicative inverse $[k]_{9}$. Then \newline
	${\left[6\right]}_{9}×{\left[k\right]}_{9}={\left[6×k\right]}_{9}={\left[1\right]}_{9}$ \newline
	Which means there exists an integer i such that\newline
	1 = 6k + 9i which implies gcd(6,9) = 1 which is not the case\newline
}

\frame{\frametitle{Equivalence Class Modulo n Containing m (cont)}
	
	This will work if we only include the relatively prime numbers for example\newline
	${z}_{9}^{\ast }=\left\{{\left[1\right]}_{1},{\left[2\right]}_{9},{\left[4\right]}_{9},{\left[5\right]}_{q},{\left[7\right]}_{9},{\left[8\right]}_{9}\right\}$\newline
	Using $({z}_{9}^{\ast },*)$ we have the following multiplicative inverses \newline
	${\left[1\right]}_{9} * {\left[1\right]}_{9}={\left[1\right]}_{9}$ \newline
	${\left[2\right]}_{9} * {\left[5\right]}_{9}={\left[10\right]}_{9}={\left[1\right]}_{9}$ \newline
	${\left[4\right]}_{9} * {\left[7\right]}_{9}={\left[28\right]}_{9}={\left[1\right]}_{9}$ \newline
	${\left[8\right]}_{9} * {\left[8\right]}_{9}={\left[64\right]}_{9}={\left[1\right]}_{9}$ \newline
	The number of elements in ${z}_{n}^{\ast }$ is given by Euler's totient function\newline
	$\phi \left(n\right)=n\prod _{p:p|n}\left(1-\frac{1}{p}\right)$
	For example
	$\phi \left(60\right)=60\prod _{p:p|60}\left(1-\frac{1}{p}\right) = 60\left(1-\frac{1}{2}\right)\left(1-\frac{1}{3}\right)\left(1-\frac{1}{5}\right)=16$\newline
	If the number is prime the totient function is simply $\phi \left(p\right)=p-1$
}


\frame{\frametitle{SubGroups}
	If G =  (S, *) is a group, S' $\subseteq $ S, and G' = (S', *) is a group then G' is said to be a subgroup of G. It is a proper subgroup if S' $\ne$ S \newline
	For E, the set of even integers and Z the set of integers. \newline
	(E, +) is a proper subgroup of (Z, +)\newline
	$\mid$S$\mid$ denotes the number of elements in S it has been shown $\mid$S'$\mid$ $\mid$ $\mid$S$\mid$ \newline\newline
	Suppose we have a finite group G = (S, *) and a $\in$ S.\newline
	$<a> = \{a^{k}$ such that k is a positive integer $\}$ \newline
	Clearly $<a>$ is closed under *. So, $(<a>,*)$ is a subgroup of G.\newline
	This new group is called the subgroup generated by a.\newline
	If the subgroup genereated by a is G we call a a generator of G \newline
	For example ($\textbf{Z}_{6} ,+$). We have \newline
	$<[2]_{6}> = \{[2]_{6}, [2]_{6} + [2]_{6}, [2]_{6} + [2]_{6} + [2]_{6}, ...\}$\newline
	\hspace*{1.4cm}$= \{[2]_{6},[4]_{6},[0]_{6},[2]_{6},... \}$	
}

\frame{\frametitle{SubGroups (cont)}
	When generating a subgroup we can stop once we reach the identity element\newline
	ord(a) is the least positive integer t such that $a^{t} = e$ where e is the identity element\newline 
	Consider the group ($\textbf{Z}_{6} ,+$). We have 
	$<[3]_{6}> = \{[3]_{6}, [3]_{6} + [3]_{6}\} = \{[3]_{6},[0]_{6}\}$\newline
	and \newline
	$<[2]_{6}> = \{[2]_{6}, [2]_{6} + [2]_{6}, [2]_{6} + [2]_{6} + [2]_{6}\} = \{[2]_{6},[4]_{6}, [0]_{6}\}$\newline
	Clearly \newline
	$<[1]_{6}> = \textbf{Z}_{6}$\newline	
}

\frame{\frametitle{SubGroups (cont)}
	Euler proved for any integer n > 1 for all $[m]_{n} \in \textbf{Z}_{n}$ \newline
	${\left({\left|m\right|}_{n}\right)}^{\phi \left(n\right)}={\left|1\right|}_{n}$ \newline
	Consider the group $(\textbf{Z}_{20}, *)$ We have that \newline
	$\phi \left(20\right)=20\prod _{p:p|20}\left(1-\frac{1}{p}\right)=20\left(1-\frac{1}{5}\right)\left(1-\frac{1}{2}\right)=8$\newline
	and ${\left({\left[3\right]}_{20}\right)}^{8}={\left[6561\right]}_{20}={\left[1\right]}_{20}$\newline
	\newline
	Also Fermat has shown that if p is prime then for all $[m]_{p} \in \textbf{Z}_{p}$ \newline
	${\left({\left[m\right]}_{p}\right)}^{p-1}={\left[1\right]}_{p}$ \newline
	For example group $(\textbf{Z}_{7}, *)$. We have that\newline
	${\left({\left[2\right]}_{7}\right)}^{7-1}={\left[64\right]}_{7}={\left[1\right]}_{7}$
}

\section{Solving Modular Linear Equations}

\frame{\frametitle{Pre Solving Modular Linear Equations}
	The modular equation \newline
	${\left[m\right]}_{n}x={\left[k\right]}_{n}$\newline
	 for X, where X is an equivalence class modulo n, and m, n $>$ 0. \newline
	 ${<\left[6\right]>}_{8}=\left\{{\left[0\right]}_{8},{\left[6\right]}_{8},{\left[4\right]}_{8},{\left[2\right]}_{8}\right\}$\newline
	 the equation \newline
	${\left[6\right]}_{8}x={\left[K\right]}_{8}$\newline
	has a solution if and only if $[k]_{8}$ is $[0]_{8},[6]_{8},[4]_{8}, or [2]_{8}$ For example, solutions to \newline
	${\left[6\right]}_{8}x={\left[4\right]}_{8}$\newline
	are x = $[2]_{8}$ and x = $[6]_{8}$
}

\frame{\frametitle{Pre Solving Modular Linear Equations (cont)}
	Consider the group $(\textbf{Z}_{n}, +)$ For any $[m]_{n} \in  \textbf{Z}_{n}$ we have that

$⟨{\left[m\right]}_{n}⟩=⟨{\left[d\right]}_{n}⟩=\left\{{\left[0\right]}_{n},{\left[d\right]}_{n},{\left[2d\right]}_{n},\dots ,{\left[\left(n∕d-1\right)d\right]}_{n}\right\}$
	
	where d = gcd(n, m). This means \newline
	$ord({\left[m\right]}_{n})=\left|<{\left[m\right]}_{n}>\right|=\frac{n}{d}$\newline
	\newline
	The equation $\left[m\right]_{n} x=\left[k\right]_{d}$ \newline
	has a solution if and only if d $\mid$ where d = gcd(n,m). 
	Furthermore if the equation has a solution it has d solutions. \newline
	There is only a solution for every equivalence class $[k]_{n}$ if and only if gcd(n,m) = 1

}


\frame{\frametitle{Pre Solving Modular Linear Equations Examples}
	Using the group $(\textbf{Z}_{8}, +)$. Since gcd(8,5) = 1 \newline
	So, ${\left[5\right]}_{8}x={\left[k\right]}_{8}$ has exactly one solution when solving for any k that is a member of $<[5]>_{8}$. When k = 3 we know that x = $[7]_{8}$ \newline
	Using the same group we use 6 instead so gcd(8,6) = 2 \newline
	So, ${\left[6\right]}_{8}x={\left[k\right]}_{8}$ has exactly two solutions when solving for any k that is a member of $<[6]>_{8}$. When k = 4 we know that x = $[6]_{8}$ and x = $[2]_{8}$ \newline
	
}


\frame{\frametitle{Solving Modular Linear Equations}
	Let d = gcd(n,m) and let i and j be integers such that d = in + jm \newline
	Suppose further d $\mid$ k Then the equation 
	${\left[m\right]}_{n}x={\left[k\right]}_{n}$ has solution \newline
	$x={\left[\frac{jk}{d}\right]}_{n}$
	For example, consider ${\left[6\right]}_{8}x={\left[4\right]}_{8}$ we have gcd(8,6) = 2 \newline
	2 = (1) 8 + (-1) 6 and 2 $\mid$ 4 so it must have the solution \newline
	$x={\left[\frac{-1\left(4\right)}{2}\right]}_{8}={\left[-2\right]}_{8}={\left[6\right]}_{\delta }$
	This is only one solution though to solve the other we use the equation \newline
	${\left[j+\frac{wn}{d}\right]}_{n}$ for $w=0,1,\dots ,d-1$ \newline
	So for the other solution we have\newline
	${\left[6+\frac{1\left(8\right)}{2}\right]}_{8}={\left[10\right]}_{8}={\left[2\right]}_{8}$
}

\frame{\frametitle{Psuedocode For Solving Modular Linear Equations}
	void solvelinear ( int n, int m, int k) \newline
	\hspace*{1cm}index l;\newline
	\hspace*{1cm}int i, j, d;\newline
	\hspace*{1cm}Euclid(n,m,d,i,j);\newline
	\hspace*{1cm}if (d$\mid$k)\newline
	\hspace*{2cm}for(w = 0; w <= d - 1; w++)\newline
	\hspace*{3cm}cout << ${\left[\frac{jk}{d}+\frac{wn}{d}\right]}_{n};$
	
	Worst Case Time Complexity is Exponential in terms of input size
}

\section{Computing Modular Powers}

\subsection{Computing Modular Powers}
\frame{\frametitle{Computing Modular Powers}
	
}

\section{Finding Large Prime Numbers}

\subsection{Searching for a Large Prime}
\frame{\frametitle{Searching for a Large Prime}
	
}

\subsection{Checking if a Number is Prime}
\frame{\frametitle{Checking if a Number is Prime}
	
}

\section{RSA Public-Key Cryptosystem}


\subsection{Public-Key Cryptosystems}
\frame{\frametitle{Public-Key Cryptosystems}
	
}

\subsection{RSA Cryptosystem}
\frame{\frametitle{RSA Cryptosystem}
	
}




\end{document}

