% This text is proprietary.
% It's a part of presentation made by myself.
% It may not used commercial.
% The noncommercial use such as private and study is free
% Sep. 2005 
% Author: Sascha Frank 
% University Freiburg 
% www.informatik.uni-freiburg.de/~frank/

\documentclass{beamer}
\usepackage{multicol}

\begin{document}
\title{Chapter 11 Cryptology}   
\author{Jason and Sam!} 
\date{\today} 

\frame{\titlepage} 

\frame{
	\begin{multicols}{2}
	\frametitle{Overview}\tableofcontents
	\end{multicols}
} 


\section{Number Theory} 

\subsection{Composite and Prime Numbers}
\frame{\frametitle{Composite and Prime Numbers} 
Composite Numbers have a divisor other than itself and one. \newline
For example 4$\mid$20 means that 20 = 5 * 4 \newline
Prime numbers have no divisors but 1 and itself \newline
First 10 Primes \newline
2, 3, 5, 7, 11, 13, 17, 19, 23, 29
}
\subsection{Greatest Common Divisor}
\frame{\frametitle{Greatest Common Divisor}
	A common divisor is a number that is a factor of both numbers \newline
	For example gcd(12,15) = 3 \newline
	The greatest common divisor is the largest factor for both numbers \newline
	Now h$\mid$n and h$\mid$m then h$\mid$(in + jm) where i and j are any constant \newline
	
}

\subsection{Prime Factorization}
\frame{\frametitle{Prime Factorization}
	Every integer X $>$ 1 can be written as a unique product of primes \newline
	That is X = p$_{1}^{k_{1}}$ * p$_{2}^{k_{2}}$ * ... * p$_{n}^{k_{n}}$  \newline
	Where p$_{1}<$p$_{2}<$...p$_{n}$ and this representation of n is unique \newline
	Example being 22,275 = 3$^{4}$ * 5$^{2}$ * 11 \newline
	There is no polynomial time algorithm for finding determining the factorization of an integer \newline
}

\subsection{Least Common Multiple}
\frame{\frametitle{Least Common Multiple}
	
}

\section{Euclid's Algorithm}

\subsection{Euclid's Algorithm}
\frame{\frametitle{Euclid's Algorithm}
	
}

\subsection{Extension to Euclid's Algorithm}
\frame{\frametitle{Extension to Euclid's Algorithm}
	
}

\section{Modular Arithmetic}

\subsection{Group Theory}
\frame{\frametitle{Group Theory}
	
}

\subsection{Congruency Modulo n}
\frame{\frametitle{Congruency Modulo n}
	
}



\subsection{Subgroups}
\frame{\frametitle{Subgroups}
	
}

\section{Solving Modular Linear Equations}

\subsection{Solving Modular Linear Equations}
\frame{\frametitle{Solving Modular Linear Equations}
	
}

\section{Computing Modular Powers}

\subsection{Computing Modular Powers}
\frame{\frametitle{Computing Modular Powers}
	
}

\section{Finding Large Prime Numbers}

\subsection{Searching for a Large Prime}
\frame{\frametitle{Searching for a Large Prime}
	
}

\subsection{Checking if a Number is Prime}
\frame{\frametitle{Checking if a Number is Prime}
	
}

\section{RSA Public-Key Cryptosystem}


\subsection{Public-Key Cryptosystems}
\frame{\frametitle{Public-Key Cryptosystems}
	
}

\subsection{RSA Cryptosystem}
\frame{\frametitle{RSA Cryptosystem}
	
}




\end{document}

