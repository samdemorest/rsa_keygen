% This text is proprietary.
% It's a part of presentation made by myself.
% It may not used commercial.
% The noncommercial use such as private and study is free
% Sep. 2005 
% Author: Sascha Frank 
% University Freiburg 
% www.informatik.uni-freiburg.de/~frank/

\documentclass{beamer}
\usepackage{multicol}
\usepackage{amsmath}

\begin{document}
\title{Chapter 11 Cryptology}   
\author{Jason Pearson and Sam Demorest} 
\date{\today} 

\frame{\titlepage} 

\frame{
	\begin{multicols}{1}
	\frametitle{Overview}\tableofcontents
	\end{multicols}
} 


\section{Number Theory Review} 

\frame{\frametitle{Composite and Prime Numbers} 
Composite Numbers have a divisor other than itself and one. \newline
For example 4$\mid$20 means that 20 = 5 * 4 \newline
The divisors of 12 are 1,2,3,4,6 and 12 \newline
Prime numbers have no divisors but 1 and itself \newline
First 10 Primes \newline
2, 3, 5, 7, 11, 13, 17, 19, 23, 29
}

\section{Greatest Common Divisor}
\frame{\frametitle{Greatest Common Divisor}
	If h$\mid$m and h$\mid$n then h is called a common divisor\newline
	A common divisor is a number that is a factor of both numbers \newline
	The greatest common divisor is the largest factor for both numbers \newline	
	This is denoted gcd(n,m) \newline
	For example gcd(12,15) = 3 \newline \newline
	For any two integers n and m where m $\ne$ 0 the quotient is n divided by n is given by \newline
	$q = \lfloor n / m \rfloor$\newline
	The remainder r of dividing n by m is given by \newline
	$r = n - qm$\newline
}

\frame{\frametitle{Greatest Common Divsor (cont)}
	Let n and m be integers, not both 0 and let\newline
	d = min \{ in + jm such that i,j $\in$ Z and in + jm > 0\}\newline
	That is, d is the smallest positive linear combination of n and m \newline
	For example we know gcd(12, 8) = 4,\newline the smallest linear combination is \newline
	$4 = 3(12) + (-4)8$
	\newline \newline
	Now suppose we have n $\ge$ 0 and m $>$ 0 and r = n mod(m) then \newline
	gcd(n , m) = gcd(m , r)\newline
	so gcd(64 , 24) = gcd(24, 16)\newline
	\hspace*{2.45cm} = gcd(16, 8) \newline
	\hspace*{2.45cm} = gcd(8, 0) \newline
	\hspace*{2.45cm} = 8 \newline
}


\frame{\frametitle{Least Common Multiple}
	For n and m where they are both nonzero, the least common multiple is denoted lcm(n,m)\newline
	For example lcm(6,9) = 18 because 6$\mid$18 and 9 $\mid$18\newline
	The lcm(n,m) is a product of primes that are common to m and n, where the power of each prime in the product is the larger of its orders in n and m\newline
	So $12={2}^{2}×{3}^{1}$ and $45={3}^{2}×{5}^{1}$\newline
	so lcm(12,45) = ${2}^{2}×{3}^{2}×{5}^{1}$ = 180
}

\frame{\frametitle{Prime Factorization}
	Two integers are relatively prime because the gcd of them is 1 \newline
	For example gcd(12, 25) = 1 so they are relatively prime \newline
	If h and m are relatively prime and h divides nm, then h divides m.\newline
	That is gcd(h,m) = 1 and h$\mid$nm implies h$\mid$n \newline
}

\frame{\frametitle{Prime Factorization (cont)}
	Every integer X $>$ 1 can be written as a unique product of primes \newline
	That is X = p$_{1}^{k_{1}}$ * p$_{2}^{k_{2}}$ * ... * p$_{n}^{k_{n}}$  \newline
	Where p$_{1}<$p$_{2}<$...p$_{n}$ and this representation of n is unique \newline
	Example being 22,275 = 3$^{4}$ * 5$^{2}$ * 11 \newline \newline
	To solve gcd(3,185,325, 7,276,500) we know \newline
	3,185,325 = ${3}^{4}{5}^{2}{11}^{2}{13}^{1}$\newline
	7,276,500 = ${2}^{2}{3}^{3}{5}^{3}{7}^{2}{11}^{1}$\newline
	We then take the common divisors and take the lower power to create the gcd \newline
	so gcd(3,185,325, 7,276,500) = ${3}^{3}{5}^{2}{11}^{1}$ = 7,425
}


\section{Euclid's Algorithm}

\subsection{Euclid's Algorithm}
\frame{\frametitle{Euclid's Algorithm}
	Euclid's Algorithm gives us a straight forward way to find the gcd of two numbers \newline
	int gcd(int n, int m)\newline
	\{\newline
	\hspace*{1cm}if(m == 0)\newline
	\hspace*{2cm} return n;\newline
	\newline
	\hspace*{1cm}else\newline
	\hspace*{2cm} return gcd(m, n mod m);\newline
	\}
}

\frame{\frametitle{Euclid's Algorithm WTC}
		
}

\subsection{Extension to Euclid's Algorithm}
\frame{\frametitle{Extension to Euclid's Algorithm}
	
}

\section{Modular Arithmetic}

\subsection{Group Theory}
\frame{\frametitle{Group Theory}
	
}

\subsection{Congruency Modulo n}
\frame{\frametitle{Congruency Modulo n}
	
}



\subsection{Subgroups}
\frame{\frametitle{Subgroups}
	
}

\section{Solving Modular Linear Equations}

\subsection{Solving Modular Linear Equations}
\frame{\frametitle{Solving Modular Linear Equations}
	
}

\section{Computing Modular Powers}

\subsection{Computing Modular Powers}
\frame{\frametitle{Computing Modular Powers}
	
}

\section{Finding Large Prime Numbers}

\subsection{Searching for a Large Prime}
\frame{\frametitle{Searching for a Large Prime}
	
}

\subsection{Checking if a Number is Prime}
\frame{\frametitle{Checking if a Number is Prime}
	
}

\section{RSA Public-Key Cryptosystem}


\subsection{Public-Key Cryptosystems}
\frame{\frametitle{Public-Key Cryptosystems}
	
}

\subsection{RSA Cryptosystem}
\frame{\frametitle{RSA Cryptosystem}
	
}




\end{document}

